\documentclass[a4paper,12pt]{article}
\usepackage[T1]{fontenc}
\usepackage[utf8]{inputenc}
\usepackage{lmodern}
\usepackage[francais]{babel}
\usepackage[francais]{babel}
\usepackage{amsmath,amsthm,amssymb,amsfonts,bm, mathtools}
\usepackage{graphicx}
\usepackage{array}
%\usepackage[colorinlistoftodos]{todonotes}
%\usepackage{comment}
\usepackage{boiboites}
\usepackage{pgf,tikz}
\usetikzlibrary{arrows}
%%%%%%%%%%%%%%%%%%%%%%%%%%%%%%%%%%%%%%%%%%%%%%%%%%%%%%%%%%%%%%%%%%%%%
%%          DEFINITIONS D'USAGE       - RACCOURCI                  %%
%%%%%%%%%%%%%%%%%%%%%%%%%%%%%%%%%%%%%%%%%%%%%%%%%%%%%%%%%%%%%%%%%%%%%
\definecolor{ududff}{rgb}{0.30196078431372547,0.30196078431372547,1.}
\definecolor{qqwuqq}{rgb}{0,0.39,0}
\definecolor{qqzzff}{rgb}{0,0.6,1}
\definecolor{uququq}{rgb}{0.25,0.25,0.25}
\definecolor{xdxdff}{rgb}{0.49,0.49,1}
\definecolor{qqqqff}{rgb}{0,0,1}
\definecolor{ttqqqq}{rgb}{0.2,0,0}
\definecolor{zzttqq}{rgb}{0.6,0.2,0}
\definecolor{cqcqcq}{rgb}{0.75,0.75,0.75}
\definecolor{cyann}{rgb}{0.79,0.88,1} % for strings
\definecolor{yel}{rgb}{1,1,0.75} % for strings
\definecolor{redd}{rgb}{1,.5,.5} % for strings
%%%%%%%%%%%%%%%%%%%%%%%%%% Grilles %%%%%%%%%%%%%%%%%%%%%%%%%%%%%
\usepackage{tikz}

%\usepackage{./packages/boites_exemples}
\usepackage{color}
\definecolor{darkblue}{rgb}{0,0,0}
\usepackage{hyperref}                 
 \hypersetup{
    hyperfigures = true,
    colorlinks = true,
    linkcolor=darkblue
    }
%pour redefinir les marges
\usepackage[top=2cm, bottom=2cm, left=2cm, right=2cm]{geometry}
%%%%%%%%%%%%%%%%%%%%%%%%%%%%%%%%%%%%%%%%%%%%%%%%%%%%%%%%%%%%%%%%%%%
%%%%%%%%%%%%%%%%% ENVIRONNEMENT PROP ET DEF PERSONNEL %%%%%%%%%%%%%
%%%%%%%%%%%%%%%%%%%%%%%%%%%%%%%%%%%%%%%%%%%%%%%%%%%%%%%%%%%%%%%%%%%

\newboxedtheorem[titlecolor = black, titlebackground = white, background = white,
titleboxcolor = black, boxcolor = black, thcounter=, size = .9\textwidth]{coro}{Corollaire}{compteurCO}

\newboxedtheorem[titlecolor = black, titlebackground = redd, background = yel,
titleboxcolor = black, boxcolor = black, thcounter=, size = .9\textwidth]{Prop}{Propriété}{compteurPR}

\newboxedtheorem[titlecolor = black, titlebackground = redd, background = yel,
titleboxcolor = black, boxcolor = black, thcounter=, size = .9\textwidth]{Propo}{Proposition}{compteurPP}

\newboxedtheorem[titlecolor = black, titlebackground = white, background = white,
titleboxcolor = black, boxcolor = black, thcounter=, size = .9\textwidth]{theo}{Théorème}{compteurTH}

\newboxedtheorem[titlecolor = black, titlebackground = white, background = white,
titleboxcolor = black, boxcolor = black, thcounter=, size = .9\textwidth]{lem}{Lemme}{compteurLE}

\newboxedtheorem[titlecolor = black, titlebackground = cyann , background = yel,
titleboxcolor = black, boxcolor = black, thcounter=, size = .9\textwidth]{Def}{Définition}{compteurDE}

%%%%%%%%%%%%%%%%%%%%% STYLE DES THEOREMES PROP ETC %%%%%%%%%%%%%%%%
\newtheoremstyle{StyleTheo_will}
{3pt}
{5pt}
{\upshape}
{}
{\scshape}
{\newline}
{0pt}
{}
%%%%%%%%%%%%%%%%%%%% STYLE DES DEFINITIONS   %%%%%%%%%%%%%%%%%%%%%%
\newtheoremstyle{StyleDef_will}
{3pt}
{5pt}
{\upshape}
{}
{\slshape \bfseries}
{\newline}
{0pt}
{}
%%%%%%%%%%%%%%%%%%%%%%%%%%%%%%%%%%%%%%%%%%%%%%%%%%%%%%%%%%%%%%%%%%%%

%%  DECLARATION DES STYLES %%%%%%%%%%%%%%%%%%%%%%%%%%%%%%%%%%%%%%%%%
\theoremstyle{StyleTheo_will}
%\newtheorem{Prop}{Propri\'et\'e}[section]
%\newtheorem{Theo}{Th\'eor\`eme}[section]
%\newtheorem{Lem}{Lemme}[section]
%\newtheorem{coro}{Corollaire}[section]


%\theoremstyle{StyleDef_will}
%\newtheorem{Def}{D\'efinition}[section]

\theoremstyle{remark}
\newtheorem*{Rem}{Remarque}
\newtheorem*{Ex}{Exemple}


%%%%%%%%%%%%%%%%%%%%%%%%%%%%%%%%%%%%%%%%%%%%%%%%%%%%%%%%%%%%%%%%%%%%%




\usepackage{fancyhdr}
\lhead{Chapitre 4 SFP-2104}
\newcommand{\argmin}{\arg\!\min}
\newcommand{\argmax}{\arg\!\max}
\newcommand\rr{\mathbb{R}}
\newcommand\nn{\mathbb{N}}
\newcommand\begit{\begin{itemize}}
\newcommand\enit{\end{itemize}}
\newcommand\deff{{\bf Définition : }}
\newcommand\propp{{\bf Proposition : }}
\newcommand\itt{\item[$\bullet$]}
\newcommand{\grille}[1]{
 \begin{center}
  \begin{tikzpicture}
    \draw [very thin, gray] (0,0) grid[step=0.5] (15.5,#1);
  \end{tikzpicture}
 \end{center}
}
\newcommand{\grillepage}{\grille{24.5}}
\newcommand{\vect}[1]{\overrightarrow{#1}}

\oddsidemargin=0pt
\topmargin=-60pt
\textwidth=450pt
\textheight=700pt

\setlength{\parindent}{0pt}
\setlength{\parskip}{0.6cm}

\title{Interférences à deux ondes par division d'amplitude}
\author{Quentin CHAUVIN}

\begin{document}

\maketitle
\tableofcontents
  
\section{Généralités}
Pour les systèmes interférentiels à différences du front d'onde, la source primaire doit être quasi ponctuelle pour qu'il y est cohérence entre les sources émises. La conséquence est que les figures d'interférences sont peu lumineuses. Pour qu'il y est interférence entre deux ondes, il faut que le déphasage en un point $M$ d'observation des interférences soit tel-que : 
\[ \varPhi = \varPhi_1 - \varPhi_2 = \frac{2\pi}{\lambda_0}\delta = \frac{2\pi}{\lambda_0}\left( [SS_1M]-[SS_2M]\right) \] Si la source primaire est déplaçée de $\vect{dS}$, les chemins optique $[SS_1M]$ et $[SS_2M]$ varient respectivement de $n\vect{\Delta S}.\vect{u_1}$ et $n\vect{\Delta S}.\vect{u_2}$ avec $\vect{u_1}$ et $\vect{u_2}$ les vecteurs unitaires portés par les directions de propagation des ondes.


La variation de phase (du déphasage) $\Delta\Phi$ est égale à : \[ \Delta\Phi =\frac{2\pi}{\lambda_0}n\vect{\Delta S }\space(\vect{u_2} - \vect{u_1}) \]

Deux cas de figures se présentent pour maintenir la variation de déphasage $\Delta\Phi = 0$ lorsque la source primaire se déplace. C'est la propriété exploitée dans les systèmes interférentiels par division du front d'onde lorsque l'on passe d'une source ponctuelle à une source fine étendue dans la direction perpendiculaire au plan des ondes interférentes. 

Cela correspond à un rayon incident unique ou à des rayons paralèlles, c'est le cas que nous allons considérer avec le système interférentiel par division d'amplitude. Le dédoublement du train d'onde issu d'un rayon incident par le biais de reflexion ou de transmission par une lame mince permet d'obtenir des interférences. Une lame mince est constitué de deux dioptres. Nous allons considérer deux types de lames minces à l'origines de franges d'interférences  : la lame mince à faces parallele et la lame mince dites en coin dont les dioptres forment un angle $\alpha$ très petit. 

\section{Localisation des franges}

\subsection{Lame mince à faces}

Les rayons d'une source étendue d'une même inclinaison $i$ par rapport à lame mince sont parallèles entre eux après reflexion ou transmission par les dioptres parallèles de la lame mince. Ces rayons incidents parallèles entre eux permettent de renforcer la figure d'interférence. Celle-ci se forment à l'infini compte tenu du parallélisme interférentes issu de la lame mince. On dit que les interférences sont \underline{localisées à l'infini}.

\subsection{Lame mince dite en coin}

Lorsque la lame mince est éclairé en incidence proche de la normale au dioptre, formant un angle $\alpha$ très faible en tre eux. Le point de rencontre des rayons issus de la reflexion ou de la transmission par la lame en coin est plus ou moins proche de celle-ci. on dit que les interférences sont \underline{localisées à l'infini}.

\section{Déphasage introduit par une lame}

Bien que la démonstration qui suit soit illustrée par le cas de la lame à face parallèle, elle reste vallable pour la lame d'épaisseur uniforme appelé "lame en coin".

\begin{center}
\begin{tikzpicture}[line cap=round,line join=round,>=triangle 45,x=1.0cm,y=1.0cm]
\draw [color=cqcqcq,dash pattern=on 1pt off 1pt, xstep=0.5cm,ystep=0.5cm] (4,-1) grid (10,5);
\clip(4,-1) rectangle (10,5);
\draw [shift={(6,1)},color=qqwuqq,fill=qqwuqq,fill opacity=0.1] (0,0) -- (270:0.47) arc (270:284.04:0.47) -- cycle;
\draw [shift={(5,3)},color=qqwuqq,fill=qqwuqq,fill opacity=0.1] (0,0) -- (270:0.47) arc (270:296.57:0.47) -- cycle;
\draw [shift={(7,3)},color=qqwuqq,fill=qqwuqq,fill opacity=0.1] (0,0) -- (-90:0.47) arc (-90:-63.43:0.47) -- cycle;
\draw [shift={(8,1)},color=qqwuqq,fill=qqwuqq,fill opacity=0.1] (0,0) -- (-90:0.47) arc (-90:-75.96:0.47) -- cycle;
\draw (10,3)-- (3,3);
\draw (3,1)-- (10,1);
\draw (3,3)-- (10,3);
\draw [dash pattern=on 3pt off 3pt] (5,3.5)-- (5,0.5);
\draw [dash pattern=on 3pt off 3pt] (6,2)-- (6,0);
\draw [dash pattern=on 3pt off 3pt] (8,2)-- (8,0);
\draw [dash pattern=on 3pt off 3pt] (7,4)-- (7,2);
\draw (4.5,5)-- (5,3);
\draw (4.77,3.92) -- (4.65,3.97);
\draw (4.77,3.92) -- (4.85,4.03);
\draw (5,3)-- (6,1);
\draw (5.54,1.93) -- (5.41,1.95);
\draw (5.54,1.93) -- (5.59,2.05);
\draw (6,1)-- (6.5,-1);
\draw (6.27,-0.08) -- (6.15,-0.03);
\draw (6.27,-0.08) -- (6.35,0.03);
\draw (6,1)-- (7,3);
\draw (6.57,2.15) -- (6.63,2.03);
\draw (6.57,2.15) -- (6.44,2.12);
\draw (6.5,2) -- (6.56,1.88);
\draw (6.5,2) -- (6.37,1.97);
\draw (7,3)-- (7.5,5);
\draw (7.29,4.16) -- (7.37,4.05);
\draw (7.29,4.16) -- (7.17,4.11);
\draw (7.25,4) -- (7.33,3.89);
\draw (7.25,4) -- (7.13,3.95);
\draw (7,3)-- (8,1);
\draw (7.54,1.93) -- (7.41,1.95);
\draw (7.54,1.93) -- (7.59,2.05);
\draw (7.46,2.07) -- (7.33,2.1);
\draw (7.46,2.07) -- (7.52,2.19);
\draw (7.61,1.78) -- (7.48,1.81);
\draw (7.61,1.78) -- (7.67,1.9);
\draw (8,1)-- (8.5,-1);
\draw (8.27,-0.08) -- (8.15,-0.03);
\draw (8.27,-0.08) -- (8.35,0.03);
\draw (8.23,0.08) -- (8.11,0.13);
\draw (8.23,0.08) -- (8.31,0.18);
\draw (8.31,-0.24) -- (8.19,-0.18);
\draw (8.31,-0.24) -- (8.39,-0.13);
\draw (8,1)-- (9,3);
\draw (9,3)-- (9.5,5);
\draw (5,3)-- (5.5,5);
\draw (5.27,4.08) -- (5.35,3.97);
\draw (5.27,4.08) -- (5.15,4.03);
\draw [dash pattern=on 3pt off 3pt] (9,4)-- (9,2);
\begin{scriptsize}
\fill [color=qqqqff] (3,1) circle (1.5pt);
\draw[color=qqqqff] (3.13,1.21) node {$A$};
\fill [color=qqqqff] (3,3) circle (1.5pt);
\draw[color=qqqqff] (3.13,3.21) node {$D$};
\fill [color=xdxdff] (5,3) circle (1.5pt);
\draw[color=xdxdff] (5.13,3.21) node {$P$};
\fill [color=uququq] (6,1) circle (1.5pt);
\draw[color=uququq] (6.13,1.21) node {$Q$};
\fill [color=xdxdff] (7,3) circle (1.5pt);
\draw[color=xdxdff] (7.13,3.21) node {$S$};
\fill [color=uququq] (8,1) circle (1.5pt);
\draw[color=uququq] (8.13,1.21) node {$U$};
\fill [color=xdxdff] (9,3) circle (1.5pt);
\draw[color=xdxdff] (9.14,3.21) node {$W$};
\draw[color=qqwuqq] (6.28,0.77) node {$\alpha$};
\end{scriptsize}
\end{tikzpicture}
\end{center}

Soit une lame mince transparentes d'indice $n$ à face parralèle d'épaisseur $e$ plongée dans un milieu d'indice $n_0$. Soit une onde plane dont on considère que l'amplitude vaut $1$, se popageant dasn le milieu d'indice $n_0$ avec une incide proche de la normal au dioptrede la lame mince. Cette onde créée une infinité d'onde réfléchie et transmise par la lame mince. Les coeficients de reflexion et de transmission $(r,t)$ pour une incidence proche de la normal sont : 
\begin{center}
  \begin{tabular}{>{\centering}m{3cm}>{\centering}m{3cm}>{\centering}m{3cm}>{\centering}m{3cm}}
    $ r_{1_{(/2)}} = \frac{n_0 - n}{n_0 + n}$ & \begin{tikzpicture}[line cap=round,line join=round,>=triangle 45,x=1.0cm,y=1.0cm]
\draw [color=cqcqcq,dash pattern=on 1pt off 1pt, xstep=1.0cm,ystep=1.0cm] (2,0) grid (5,4);
\clip(2,0) rectangle (5,4);
\draw (2,2)-- (5,2);
\draw (2,4)-- (3.5,2);
\draw (2.85,2.87) -- (2.72,2.87);
\draw (2.85,2.87) -- (2.88,3);
\draw (2.75,3) -- (2.62,3);
\draw (2.75,3) -- (2.78,3.13);
\draw (3.5,2)-- (5,4);
\draw (4.35,3.13) -- (4.38,3);
\draw (4.35,3.13) -- (4.22,3.13);
\draw (4.25,3) -- (4.28,2.87);
\draw (4.25,3) -- (4.12,3);
\draw (2,3.18) node[anchor=north west] {1};
\draw (2,1.19) node[anchor=north west] {2};
\draw (3.25,3.95) node[anchor=north west] {$r_{1_{(/2)}}$};
\end{tikzpicture} & $ t_{1_{(/2)}} = \frac{2n_0}{n_0 + n}$ & \begin{tikzpicture}[line cap=round,line join=round,>=triangle 45,x=1.0cm,y=1.0cm]
\draw [color=cqcqcq,dash pattern=on 1pt off 1pt, xstep=1.0cm,ystep=1.0cm] (2,0) grid (5,4);
\clip(2,0) rectangle (5,4);
\draw (2,2)-- (5,2);
\draw (2,4)-- (3.5,2);
\draw (2.85,2.87) -- (2.72,2.87);
\draw (2.85,2.87) -- (2.88,3);
\draw (2.75,3) -- (2.62,3);
\draw (2.75,3) -- (2.78,3.13);
\draw (2,3.18) node[anchor=north west] {1};
\draw (2,1.19) node[anchor=north west] {2};
\draw (3.5,2)-- (4,0);
\draw (3.79,0.84) -- (3.67,0.9);
\draw (3.79,0.84) -- (3.87,0.95);
\draw (3.75,1) -- (3.63,1.05);
\draw (3.75,1) -- (3.83,1.1);
\draw (3.25,3.95) node[anchor=north west] {$t_{1_{(/2)}}$};
\end{tikzpicture} \tabularnewline 
  \end{tabular}
\end{center}
Avec $n_0$ l'indice du milieu $1$ et $n$ celui du milieu $2$.
\begin{center}
  \begin{tabular}{>{\centering}m{3cm}>{\centering}m{3cm}>{\centering}m{3cm}>{\centering}m{3cm}}
    $ r_{2_{(/1)}} = \frac{n - n_0}{n_0 + n}$ & \begin{tikzpicture}[line cap=round,line join=round,>=triangle 45,x=1.0cm,y=1.0cm]
\draw [color=cqcqcq,dash pattern=on 1pt off 1pt, xstep=1.0cm,ystep=1.0cm] (2,0) grid (5,4);
\clip(2,0) rectangle (5,4);
\draw (2,2)-- (5,2);
\draw (2,3.18) node[anchor=north west] {1};
\draw (2,1.19) node[anchor=north west] {2};
\draw (3.5,2)-- (5,0);
\draw (4.35,0.87) -- (4.22,0.87);
\draw (4.35,0.87) -- (4.38,1);
\draw (4.25,1) -- (4.12,1);
\draw (4.25,1) -- (4.28,1.13);
\draw (3.25,3.95) node[anchor=north west] {$r_{2_{(/1)}}$};
\draw (2,0)-- (3.5,2);
\draw (2.85,1.13) -- (2.88,1);
\draw (2.85,1.13) -- (2.72,1.13);
\draw (2.75,1) -- (2.78,0.87);
\draw (2.75,1) -- (2.62,1);
\end{tikzpicture}  & $ t_{2_{(/1)}} = \frac{2n}{n_0 + n}$ & \begin{tikzpicture}[line cap=round,line join=round,>=triangle 45,x=1.0cm,y=1.0cm]
\draw [color=cqcqcq,dash pattern=on 1pt off 1pt, xstep=1.0cm,ystep=1.0cm] (2,0) grid (5,4);
\clip(2,0) rectangle (5,4);
\draw (2,2)-- (5,2);
\draw (3.5,2)-- (4,4);
\draw (3.79,3.16) -- (3.87,3.05);
\draw (3.79,3.16) -- (3.67,3.1);
\draw (3.75,3) -- (3.83,2.9);
\draw (3.75,3) -- (3.63,2.95);
\draw (2,3.18) node[anchor=north west] {1};
\draw (2,1.19) node[anchor=north west] {2};
\draw (3.28,0.82) node[anchor=north west] {$t_{2_{(/1)}}$};
\draw (2,0)-- (3.5,2);
\draw (2.85,1.13) -- (2.88,1);
\draw (2.85,1.13) -- (2.72,1.13);
\draw (2.75,1) -- (2.78,0.87);
\draw (2.75,1) -- (2.62,1);
\end{tikzpicture}
  \end{tabular}
\end{center}

Deux relations sont à prendre en compte entre ces coefficients. La première est immédiate : $r_1 = -r_2$. La deuxième est issu du théorème de conservation de l'énergie en un point du dioptre : $r+t = 1$ avec $r$ et $t$ les coefficients de réflexion et de transmission en énergie.

Appliqué au dioptre A les coefficients en énergie pour une incidence proche de la normale sont : 
\[ R_{1_{(/2)}} = r_1\verb||^2 ; T_{1_{(/2)}} =T_{1_{(/2)}}\verb||^2\frac{n}{n_0}\]

On en déduit donc :

 \[ r_1\verb||^2 + T_{1_{(/2)}}\verb||^2\frac{n}{n_0} = 1\]

Considérons maintenant le cas ou $n_0<n$, les deux relations précedentes donnent : 
\[r_1 = r_2 = -r < 0 \]
\[ t_1t_2 = 1 - r_1\verb||^2 = 1 - r^2 \]
Les amplitudes des 4 premières ondes réfléchies sont :

\begin{center}
  \begin{tabular}{r c l}
    $r_1$               & $\Rightarrow$ & $-r$ \\
    $t_1t_2r_2$         & $\Rightarrow$ & $(1 - r^2)r$ \\
    $t_1t_2r_2\verb||^3$& $\Rightarrow$ & $(1 - r^2)r^3$\\
    $t_1t_2r_2\verb||^5$& $\Rightarrow$ & $(1 - r^2)r^5$
  \end{tabular}
\end{center}

Les amplitudes des 4 premières ondes transmise sont : 

\begin{center}
  \begin{tabular}{r c l}
    $t_1t_2$            & $\Rightarrow$ & $1 - r^2)$ \\
    $t_1t_2r_2\verb||^2$& $\Rightarrow$ & $(1 - r^2)r^2$ \\
    $t_1t_2r_2\verb||^4$& $\Rightarrow$ & $(1 - r^2)r^4$\\
    $t_1t_2r_2\verb||^6$& $\Rightarrow$ & $(1 - r^2)r^6$
  \end{tabular}
\end{center}

Dans le cas de dioptre non traités pour augmenter leurcoefficients de reflexion $r \ll 1$ (Ex  : si la lame est en verre d'indice $n = 1,5$ baignant dans un milieu d'indice $n_0 = 1$ alors $r = 0,2$ ainsi seules les deux premières ondes réfléchies et transmises ont des amplitudes non négligeables) l'études peut alors se limiter à ces deux ondes. \\ Par ailleurs les deux ondes réfléchies ayant des amplitudes beaucoup plus proches que les ondes transmises le contraste obtenu par reflexion est meilleur que celui par transmission.

\begin{tikzpicture}[line cap=round,line join=round,>=triangle 45,x=1.0cm,y=1.0cm]
\draw [color=cqcqcq,dash pattern=on 1pt off 1pt, xstep=0.5cm,ystep=0.5cm] (3,-1) grid (11,5);
\draw [color=cqcqcq,dash pattern=on 1pt off 1pt] (-1,3)  (11,5);
\clip(3,-1) rectangle (11,5);
\draw [shift={(6,1)},color=qqwuqq,fill=qqwuqq,fill opacity=0.1] (0,0) -- (270:0.46) arc (270:284.04:0.46) -- cycle;
\draw [shift={(5,3)},color=qqwuqq,fill=qqwuqq,fill opacity=0.1] (0,0) -- (270:0.46) arc (270:296.57:0.46) -- cycle;
\draw [shift={(7,3)},color=qqwuqq,fill=qqwuqq,fill opacity=0.1] (0,0) -- (-90:0.46) arc (-90:-63.43:0.46) -- cycle;
\draw [shift={(8,1)},color=qqwuqq,fill=qqwuqq,fill opacity=0.1] (0,0) -- (-90:0.46) arc (-90:-75.96:0.46) -- cycle;
\draw (3,1)-- (10,1);
\draw (10,3)-- (3,3);
\draw (3,1)-- (10,1);
\draw (3,3)-- (10,3);
\draw [dash pattern=on 3pt off 3pt] (5,3.5)-- (5,0.5);
\draw [dash pattern=on 3pt off 3pt] (6,2)-- (6,0);
\draw [dash pattern=on 3pt off 3pt] (8,2)-- (8,0);
\draw [dash pattern=on 3pt off 3pt] (7,4)-- (7,2);
\draw (4.5,5)-- (5,3);
\draw (4.77,3.92) -- (4.65,3.97);
\draw (4.77,3.92) -- (4.85,4.03);
\draw (5,3)-- (6,1);
\draw (5.54,1.93) -- (5.41,1.95);
\draw (5.54,1.93) -- (5.59,2.05);
\draw (6,1)-- (6.5,-1);
\draw (6.27,-0.08) -- (6.15,-0.03);
\draw (6.27,-0.08) -- (6.35,0.03);
\draw (6,1)-- (7,3);
\draw (6.57,2.14) -- (6.63,2.03);
\draw (6.57,2.14) -- (6.44,2.12);
\draw (6.5,2) -- (6.56,1.88);
\draw (6.5,2) -- (6.37,1.97);
\draw (7,3)-- (7.5,5);
\draw (7.29,4.16) -- (7.37,4.05);
\draw (7.29,4.16) -- (7.17,4.1);
\draw (7.25,4) -- (7.33,3.9);
\draw (7.25,4) -- (7.13,3.95);
\draw (7,3)-- (8,1);
\draw (7.54,1.93) -- (7.41,1.95);
\draw (7.54,1.93) -- (7.59,2.05);
\draw (7.46,2.07) -- (7.34,2.1);
\draw (7.46,2.07) -- (7.52,2.19);
\draw (7.61,1.78) -- (7.48,1.81);
\draw (7.61,1.78) -- (7.66,1.9);
\draw (8,1)-- (8.5,-1);
\draw (8.27,-0.08) -- (8.15,-0.03);
\draw (8.27,-0.08) -- (8.35,0.03);
\draw (8.23,0.08) -- (8.11,0.13);
\draw (8.23,0.08) -- (8.31,0.18);
\draw (8.31,-0.23) -- (8.19,-0.18);
\draw (8.31,-0.23) -- (8.39,-0.13);
\draw (8,1)-- (9,3);
\draw (9,3)-- (9.5,5);
\draw (5,3)-- (5.5,5);
\draw (5.27,4.08) -- (5.35,3.97);
\draw (5.27,4.08) -- (5.15,4.03);
\draw [dash pattern=on 3pt off 3pt] (9,4)-- (9,2);
\draw [->] (10.5,2) -- (10.5,3);
\draw [->] (10.5,2) -- (10.5,1);
\draw (10.5,2.3) node[anchor=north west] {$e$};
\draw (9.8,4.3) node[anchor=north west] {$1$};
\draw (9.8,2.3) node[anchor=north west] {$2$};
\draw (9.8,-0.02) node[anchor=north west] {$1$};
\draw (3.15,2.2) node[anchor=north west] {$n$};
\draw (3.06,4.3) node[anchor=north west] {$n_0$};
\draw (3.06,0.11) node[anchor=north west] {$n_0$};
\draw (7,3)-- (5.12,3.47);
\draw (5.24,4) node[anchor=north west] {$H$};
\draw (4.3,3) node[anchor=north west] {$I_1$};
\draw (6.2,3) node[anchor=north west] {$I_2$};
\draw (5.3,1) node[anchor=north west] {$J_1$};
\draw (7.3,1) node[anchor=north west] {$J_2$};
\draw (5,3)-- (7,3);
\draw (6,3.07) -- (6,2.93);
\draw (5,3)-- (7,3);
\draw (5.5,3) node[anchor=north west] {$K$};
\begin{scriptsize}
\fill [color=xdxdff] (5,3) circle (1.5pt);
\fill [color=xdxdff] (7,3) circle (1.5pt);
\end{scriptsize}
\end{tikzpicture}

Diférence de marche géométrique :
\[\delta_r = [I_1J_1I_1] - [I_2H] \]
\[\delta_r = n\overline{IJ_1}  + n\overline{J_1I_2} - n_0\overline{I_2H}\]
\[I_1J_1 = \frac{e}{\cos r}\]
\[I_1H = I_1I_2 \sin r\]
\[I_1I_2 = 2I_1K = 2e \times \tan i\]
\[I_1H = 2e \frac{\sin i}{\cos i} \sin r\]
Il vient : 
\[\delta_1 = \frac{2ne}{\cos r} - 2n_0e\frac{\sin i}{\cos r}\sin r\]
Or : $n_0 \sin i = n \sin r$
\[\delta_1 = \frac{2ne}{\cos r} (1- \sin^2 r) = 2ne \cos r\]

Différence de marche physique :

Si $n_0<n$ : \[ \varPhi_r = \frac{2\pi}{\lambda_0}\delta = \frac{\lambda}{2}\]
Si $n_0>n$ : \[ \varPhi_r = \frac{2\pi}{\lambda_0}\delta = \frac{\lambda}{2}\]

Anisi que nous l'avons vu précédement, à la différence de marche géométrique il faut ajouter la différence de marche liée à un milieu plus réfrungeant. DOnc :

\[\varPhi_r = \frac{2\pi}{\lambda_0}\delta \pm \pi= \frac{2\pi}{\lambda_0}2ne \cos r \pm \pi\]

\begin{center}
\begin{tikzpicture}[line cap=round,line join=round,>=triangle 45,x=1.0cm,y=1.0cm]
\draw [color=cqcqcq,dash pattern=on 1pt off 1pt, xstep=0.5cm,ystep=0.5cm] (3,-1) grid (10,5);
\clip(3,-1) rectangle (10,5);
\draw [shift={(-6,-1)},color=qqwuqq,fill=qqwuqq,fill opacity=0.15] (0,0) -- (270:0.37) arc (270:284.04:0.37) -- cycle;
\draw [shift={(-15,-9)},color=qqwuqq,fill=qqwuqq,fill opacity=0.15] (0,0) -- (270:0.37) arc (270:296.57:0.37) -- cycle;
\draw [shift={(-6,-1)},color=qqwuqq,fill=qqwuqq,fill opacity=0.15] (0,0) -- (270:0.37) arc (270:284.04:0.37) -- cycle;
\draw [shift={(-15,-9)},color=qqwuqq,fill=qqwuqq,fill opacity=0.15] (0,0) -- (270:0.37) arc (270:296.57:0.37) -- cycle;
\draw [shift={(14,6)},color=qqwuqq,fill=qqwuqq,fill opacity=0.15] (0,0) -- (-90:0.37) arc (-90:-63.43:0.37) -- cycle;
\draw [shift={(8,1)},color=qqwuqq,fill=qqwuqq,fill opacity=0.15] (0,0) -- (-90:0.37) arc (-90:-75.96:0.37) -- cycle;

\draw (10,3)-- (3,3);
\draw (3,1)-- (10,1);
\draw (3,3)-- (10,3);
\draw [dash pattern=on 3pt off 3pt] (5,3.5)-- (5,0.5);
\draw [dash pattern=on 3pt off 3pt] (6,2)-- (6,0);
\draw [dash pattern=on 3pt off 3pt] (8,2)-- (8,0);
\draw [dash pattern=on 3pt off 3pt] (7,4)-- (7,2);
\draw (4.5,5)-- (5,3);
\draw (4.77,3.94) -- (4.67,3.98);
\draw (4.77,3.94) -- (4.83,4.02);
\draw (5,3)-- (6,1);
\draw (5.53,1.94) -- (5.43,1.96);
\draw (5.53,1.94) -- (5.57,2.04);
\draw (6,1)-- (6.5,-1);
\draw (6.27,-0.06) -- (6.17,-0.02);
\draw (6.27,-0.06) -- (6.33,0.02);
\draw (6,1)-- (7,3);
\draw (6.56,2.11) -- (6.6,2.02);
\draw (6.56,2.11) -- (6.45,2.09);
\draw (6.5,2) -- (6.55,1.91);
\draw (6.5,2) -- (6.4,1.98);
\draw (7,3)-- (7.5,5);
\draw (7.28,4.12) -- (7.35,4.04);
\draw (7.28,4.12) -- (7.19,4.08);
\draw (7.25,4) -- (7.31,3.92);
\draw (7.25,4) -- (7.15,3.96);
\draw (7,3)-- (8,1);
\draw (7.53,1.94) -- (7.43,1.96);
\draw (7.53,1.94) -- (7.57,2.04);
\draw (7.47,2.06) -- (7.37,2.08);
\draw (7.47,2.06) -- (7.52,2.15);
\draw (7.59,1.83) -- (7.48,1.85);
\draw (7.59,1.83) -- (7.63,1.92);
\draw (8,1)-- (8.5,-1);
\draw (8.27,-0.06) -- (8.17,-0.02);
\draw (8.27,-0.06) -- (8.33,0.02);
\draw (8.23,0.06) -- (8.14,0.1);
\draw (8.23,0.06) -- (8.3,0.14);
\draw (8.3,-0.19) -- (8.2,-0.14);
\draw (8.3,-0.19) -- (8.36,-0.1);
\draw (8,1)-- (9,3);
\draw (9,3)-- (9.5,5);
\draw (5,3)-- (5.5,5);
\draw (5.27,4.06) -- (5.33,3.98);
\draw (5.27,4.06) -- (5.17,4.02);
\draw [dash pattern=on 3pt off 3pt] (9,4)-- (9,2);
\draw (3.23,2.38) node[anchor=north west] {$n$};
\draw (3.18,3.96) node[anchor=north west] {$n_0$};
\draw (3.18,0.12) node[anchor=north west] {$n_0$};
\draw (4.4,3.6) node[anchor=north west] {$I_1$};
\draw (5.4,0.88) node[anchor=north west] {$J_1$};
\draw (7.2,3.55) node[anchor=north west] {$I_2$};
\draw (8.38,0.9) node[anchor=north west] {$J_2$};
\draw (5.12,3.47)-- (7,3);
\draw (5.39,3.98) node[anchor=north west] {$H$};
\draw (6.12,0.53)-- (8,1);
\draw (6.31,0.58) node[anchor=north west] {$H'$};
\end{tikzpicture}
\end{center}
\[\delta_t  = [J_1I_1J_2]-[J_1H'] = 2ne \cos r\]
\[\delta_r = [I_1J_1I_2]-[I_1H]\]
\[\varPhi_t = \frac{2\pi}{\lambda_0}\]

SI $n_0<n$ il n'y a aucun déphasage lié à la nature des dioptres pour les deux ondes transmises.\\ Si $n_0>n$ un premier déphasage de $\pi$ se produit en $J_1$ et un deuxième en $I_2$ pour l'onde transmise en $J_2$. Dans les deux cas les deux ondes transmises ne présentent pas de déphasage lié à la nature des dioptres. \\ EN raison du déphasage de $\pi$ entre $\varPhi_r$ et $\varPhi_t$ la figure d'interférence obtenue par reflexion est complémentaire de celle obtenue par transmission. Il y a inversion des franges sombres et des franges brillantes en passant de l'une à l'autre.
\section{Franges d'égales inclinaison : Franges d'Haidinger}

Cette figure d'interférence est obtenue à partir d'une lame mince à face parallèle éclairée par une source monochromatique étendue. Les deux ondes issues de la reflexion ou de la transmisson par la lame ne sont pas de même amplitude, donc pas de même intensité. L'expression des intensités vu dans le Chapitre 3 : 

\[I(M) = I_1 I_2 + \sqrt{I_1I_2}\cos \varTheta 2 \cos \varPhi\]

Avec $\varTheta$ l'angle entre les champs électrique issu des deux sources secondaires. Or ces ondes provenent de la reflexion ou de la transmission par une lame à faces parallèles sont parallèles entre elles ; $\varTheta = 0$ et donc $\cos \varTheta = 1$.

\[I(M) = I_1 I_2 + \sqrt{I_1I_2} 2 \cos \varPhi\]
De manière générale :
\[\varPhi = \frac{2\pi}{\lambda_0}2ne \cos r + \varPhi_{phi}\]
\[\varPhi = \frac{2\pi}{\lambda_0}(2ne \cos r + \delta_{phi})\]
De ces deux epressions de $I$ et de $\varPhi$ on en déduit que les points dégales intensité sont ceux pour lequels le déphasage $\varPhi$ est identique, c'est à dire pour une lame d'épaisseur constante $e$ ceux pour lequels $r$ est égal à une constante, donc $i$ est égal à une constante. Cela justifie le nombre de franges à égales inclinaison puisqu'une frange est obtenue à partir de l'ensemble des ondes issus de la source étendu et présentant la même incidence sur la lame.


INSERER FIGURE.

Soit deux exemples d'observation de franges d'égales inclinaison : en transmission et en reflexion ; dans les deux cas la figure d'interférence est constitué d'anneaux concentriques localisés à l'infini puisqu'elle provient des interférences produite par des rayons parallèle entre eux.

\subsection{Iterféromètre de Michelson : utilisation en lame d'air}

L'interféromètre de Michelson est constitué de deux mirroirs $M_1$ et $M_2$, d'une lame en verre avec une face métalisé pour être semi réflechissante appellée lame séparatrice $L_s$ et d'une lame transparente de même indice et de même épaisseur appelée lame compensatrice $L_c$. Celle-ci permet de compenser le nombre de fois ou deux rayons issu de la source et réflechi par $M_1$ et $M_2$ traversent une même épaisseur de verre

\begin{center}
\begin{tikzpicture}[line cap=round,line join=round,>=triangle 45,x=1.0cm,y=1.0cm]
\draw [color=cqcqcq,dash pattern=on 1pt off 1pt, xstep=0.5cm,ystep=0.5cm] (2.49,2.49) grid (8,7.5);
\clip(2.5,2.5) rectangle (8,7.5);
\fill[color=zzttqq,fill=zzttqq,fill opacity=0.1] (3.5,7) -- (6.1,7) -- (6.1,7.3) -- (3.5,7.3) -- cycle;
\fill[color=zzttqq,fill=zzttqq,fill opacity=0.1] (7,5.6) -- (7.3,5.6) -- (7.3,3.2) -- (7,3.2) -- cycle;
\draw [color=qqwuqq] (4,3)-- (6,5);
\draw [color=qqwuqq] (6,5)-- (5.8,5.2);
\draw [line width=1.2pt,color=qqwuqq] (5.8,5.2)-- (3.8,3.2);
\draw [color=qqwuqq] (3.8,3.2)-- (4,3);
\draw [color=qqwuqq] (4.1,4.9)-- (3.9,5.1);
\draw [color=qqwuqq] (3.9,5.1)-- (5.4,6.6);
\draw [color=qqwuqq] (5.4,6.6)-- (5.6,6.4);
\draw [color=qqwuqq] (5.6,6.4)-- (4.1,4.9);
\draw (3,4.4)-- (5,4.4);
\draw (4.06,4.4) -- (4,4.32);
\draw (4.06,4.4) -- (4,4.48);
\draw (5.3,4.3)-- (5,4.4);
\draw (5,4.4)-- (5.1,4.1);
\draw (5.1,4.1)-- (5.1,2.6);
\draw (5.1,3.29) -- (5.02,3.35);
\draw (5.1,3.29) -- (5.18,3.35);
\draw (5.1,3.41) -- (5.02,3.47);
\draw (5.1,3.41) -- (5.18,3.47);
\draw (5.1,3.17) -- (5.02,3.23);
\draw (5.1,3.17) -- (5.18,3.23);
\draw (5,4.4)-- (5,5.8);
\draw (5,5.22) -- (5.08,5.16);
\draw (5,5.22) -- (4.92,5.16);
\draw (5,5.1) -- (5.08,5.04);
\draw (5,5.1) -- (4.92,5.04);
\draw (5,5.8)-- (4.9,6.1);
\draw (5.4,6.6)-- (5.6,6.4);
\draw [line width=1.2pt,color=ttqqqq] (3.5,7)-- (6.1,7);
\draw [color=zzttqq] (6.1,7)-- (6.1,7.3);
\draw [color=zzttqq] (6.1,7.3)-- (3.5,7.3);
\draw [color=zzttqq] (3.5,7.3)-- (3.5,7);
\draw [color=zzttqq] (7,5.6)-- (7.3,5.6);
\draw [color=zzttqq] (7.3,5.6)-- (7.3,3.2);
\draw [color=zzttqq] (7.3,3.2)-- (7,3.2);
\draw [line width=1.2pt,color=ttqqqq] (7,3.2)-- (7,5.6);
\draw (7.02,6.24) node[anchor=north west] {$M_1$};
\draw (6.29,7.16) node[anchor=north west] {$M_2$};
\draw (5.8,6.54) node[anchor=north west] {$L_C$};
\draw (5.99,5.69) node[anchor=north west] {$L_S$};
\draw (5.3,4.3)-- (6.2,4.3);
\draw (5.81,4.3) -- (5.75,4.22);
\draw (5.81,4.3) -- (5.75,4.38);
\draw (7,4.3)-- (6.1,4.3);
\draw (6.49,4.3) -- (6.55,4.38);
\draw (6.49,4.3) -- (6.55,4.22);
\draw (4.9,7)-- (4.9,6.5);
\draw (4.9,6.63) -- (4.82,6.69);
\draw (4.9,6.63) -- (4.98,6.69);
\draw (4.9,6.75) -- (4.82,6.81);
\draw (4.9,6.75) -- (4.98,6.81);
\draw (4.9,6.1)-- (4.9,6.6);
\draw (4.9,6.47) -- (4.98,6.41);
\draw (4.9,6.47) -- (4.82,6.41);
\draw (4.9,6.35) -- (4.98,6.29);
\draw (4.9,6.35) -- (4.82,6.29);
\draw (3.92,4.4)-- (4.62,4.4);
\draw (4.39,4.4) -- (4.33,4.33);
\draw (4.39,4.4) -- (4.33,4.48);
\draw (4.27,4.4) -- (4.2,4.33);
\draw (4.27,4.4) -- (4.2,4.48);
\begin{scriptsize}
\fill [color=qqqqff] (3,4.4) circle (1.5pt);
\draw[color=qqqqff] (2.94,4.77) node {$S$};
\end{scriptsize}
\end{tikzpicture}
\end{center}
Pour la compréhension des phénomènes, ont ont faire abstraction de l'épaisseu $e$ de la séparatrice et donc de la compensatrice également.

Quand l'interféromètre de Michelson éclairé par une source large est réglé en lame à faces parallèles, les interférences sont localisés à l'infini et necessite donc de placer le plan d'observation dans le plan focal d'une lentille convergeante.

Si le mirroir $M_1$ est remplacé par le mirroir $M_2$ par rapport au plan de la séparatrice (soit $M'_2$) la différence de marche $\delta$ entre les deux rayons est nulle.

Ansi tout ce passe comme si les interférences étaient produites par une lame d'air entre le mirroir réel $M_2$ et le mirroir virtuel $M'_2$.
\begin{center}
\begin{tikzpicture}[line cap=round,line join=round,>=triangle 45,x=1.0cm,y=1.0cm]
\draw [color=cqcqcq,, xstep=1.0cm,ystep=1.0cm] (2.5,0.) grid (12.5,7.);
\clip(2.5,0.) rectangle (12.5,7.);
\draw[line width=0.8pt,color=qqwuqq,fill=qqwuqq,fill opacity=0.10000000149011612] (6.523068488291529,2.9538630234169423) -- (6.769205464874587,3.0769315117084712) -- (6.646136976583058,3.323068488291529) -- (6.4,3.2) -- cycle; 
\draw [shift={(7.,6.)},line width=0.8pt,color=qqwuqq,fill=qqwuqq,fill opacity=0.10000000149011612] (0,0) -- (270.:0.3891767311949958) arc (270.:296.565051177078:0.3891767311949958) -- cycle;
\draw [shift={(8.,4.)},line width=0.8pt,color=qqwuqq,fill=qqwuqq,fill opacity=0.10000000149011612] (0,0) -- (270.:0.3891767311949958) arc (270.:296.565051177078:0.3891767311949958) -- cycle;
\draw [shift={(6.,4.)},line width=0.8pt,color=qqwuqq,fill=qqwuqq,fill opacity=0.10000000149011612] (0,0) -- (-90.:0.3891767311949958) arc (-90.:-63.43494882292201:0.3891767311949958) -- cycle;
\draw [line width=0.8pt] (3.,6.)-- (11.,6.);
\draw [line width=0.8pt] (3.,4.)-- (11.,4.);
\draw [line width=0.8pt,color=ududff] (4.,0.)-- (6.,4.);
\draw [line width=0.8pt,color=ududff] (5.030457896917461,2.0609157938349214) -- (5.078320306359187,1.9608398468204065);
\draw [line width=0.8pt,color=ududff] (5.030457896917461,2.0609157938349214) -- (4.921679693640815,2.039160153179592);
\draw [line width=0.8pt,color=ududff] (8.,4.)-- (10.,0.);
\draw [line width=0.8pt,color=ududff] (9.060915793834923,1.8781684123301554) -- (8.952137590558277,1.8999240529854848);
\draw [line width=0.8pt,color=ududff] (9.060915793834923,1.8781684123301554) -- (9.108778203276646,1.9782443593446704);
\draw [line width=0.8pt,color=ududff] (9.,2.) -- (8.891221796723356,2.021755640655329);
\draw [line width=0.8pt,color=ududff] (9.,2.) -- (9.047862409441725,2.1000759470145147);
\draw [line width=0.8pt,dash pattern=on 3pt off 3pt,color=ududff] (6.,4.)-- (7.,6.);
\draw [line width=0.8pt,dash pattern=on 3pt off 3pt,color=ududff] (6.560915793834923,5.121831587669845) -- (6.608778203276647,5.02175564065533);
\draw [line width=0.8pt,dash pattern=on 3pt off 3pt,color=ududff] (6.560915793834923,5.121831587669845) -- (6.452137590558276,5.100075947014515);
\draw [line width=0.8pt,dash pattern=on 3pt off 3pt,color=ududff] (6.5,5.) -- (6.547862409441725,4.899924052985486);
\draw [line width=0.8pt,dash pattern=on 3pt off 3pt,color=ududff] (6.5,5.) -- (6.391221796723354,4.9782443593446715);
\draw [line width=0.8pt,dash pattern=on 3pt off 3pt,color=ududff] (7.,6.)-- (8.,4.);
\draw [line width=0.8pt,dash pattern=on 3pt off 3pt,color=ududff] (7.560915793834923,4.878168412330156) -- (7.4521375905582765,4.899924052985485);
\draw [line width=0.8pt,dash pattern=on 3pt off 3pt,color=ududff] (7.560915793834923,4.878168412330156) -- (7.608778203276647,4.978244359344671);
\draw [line width=0.8pt,dash pattern=on 3pt off 3pt,color=ududff] (7.5,5.) -- (7.391221796723355,5.02175564065533);
\draw [line width=0.8pt,dash pattern=on 3pt off 3pt,color=ududff] (7.5,5.) -- (7.547862409441725,5.100075947014515);
\draw [line width=0.8pt,color=ududff] (6.,4.)-- (8.,0.);
\draw [line width=0.8pt,color=ududff] (7.030457896917461,1.939084206165078) -- (6.921679693640815,1.9608398468204071);
\draw [line width=0.8pt,color=ududff] (7.030457896917461,1.939084206165078) -- (7.078320306359187,2.039160153179593);
\draw [line width=0.8pt,dotted] (6.,5.)-- (6.,3.);
\draw [line width=0.8pt,dotted] (7.,7.)-- (7.,5.);
\draw [line width=0.8pt,dotted] (8.,5.)-- (8.,3.);
\draw [line width=0.8pt] (6.4,3.2)-- (8.,4.);
\draw [->,line width=0.8pt] (12.,5.) -- (12.,6.);
\draw [->,line width=0.8pt] (12.,5.) -- (12.,4.);
\draw (12.168619616194833,5.332486959164558) node[anchor=north west] {$e$};
\draw (3.0618841062319317,5.332486959164558) node[anchor=north west] {$n$};
\draw (3.0489115485254317,6.603797614401541) node[anchor=north west] {$n_0$};
\draw (3.0489115485254317,3.8276702652105783) node[anchor=north west] {$n_0$};
\draw (10.300571306458854,6.720550633760039) node[anchor=north west] {$M'_2$};
\draw (10.378406652697853,4.061176303927575) node[anchor=north west] {$M_1$};
\draw (5.513697512760405,4.644941400720067) node[anchor=north west] {$I$};
\draw (7.1612123414858875,6.5000171527495425) node[anchor=north west] {$J$};
\draw (8.082263938647378,4.644941400720067) node[anchor=north west] {$K$};
\draw (6.810953283410392,3.3217405146570855) node[anchor=north west] {$H$};
\draw (8.056318823234378,0.5845308385856223) node[anchor=north west] {$1$};
\draw (10.080037825448356,0.6234485117051217) node[anchor=north west] {$2$};
\draw [shift={(7.,6.)},line width=0.8pt,color=qqwuqq] (270.:0.3891767311949958) arc (270.:296.565051177078:0.3891767311949958);
\draw[line width=0.8pt,color=qqwuqq] (7.080473041541013,5.659110725676105) -- (7.0983559396612375,5.583357553604127);

\draw [shift={(7.,6.)},line width=0.8pt,color=qqwuqq] (244.:0.3891767311949958) arc (244.:270.565051177078:0.3891767311949958);
\draw[line width=0.8pt,color=qqwuqq] (7.080473041541013,5.659110725676105) -- (7.0983559396612375,5.583357553604127);

\draw [shift={(8.,4.)},line width=0.8pt,color=qqwuqq] (244.:0.3891767311949958) arc (244.:270.565051177078:0.3891767311949958);
\draw[line width=0.8pt,color=qqwuqq] (8.080473041541014,3.659110725676104) -- (8.098355939661237,3.5833575536041278);

\draw [shift={(8.,4.)},line width=0.8pt,color=qqwuqq] (270.:0.3891767311949958) arc (270.:296.565051177078:0.3891767311949958);
\draw[line width=0.8pt,color=qqwuqq] (8.080473041541014,3.659110725676104) -- (8.098355939661237,3.5833575536041278);

\draw [shift={(6.,4.)},line width=0.8pt,color=qqwuqq] (-90.:0.3891767311949958) arc (-90.:-63.43494882292201:0.3891767311949958);
\draw[line width=0.8pt,color=qqwuqq] (6.080473041541014,3.659110725676104) -- (6.098355939661239,3.5833575536041278);

\draw [shift={(6.,4.)},line width=0.8pt,color=qqwuqq] (244.:0.3891767311949958) arc (244.:270.43494882292201:0.3891767311949958);
\draw[line width=0.8pt,color=qqwuqq] (6.080473041541014,3.659110725676104) -- (6.098355939661239,3.5833575536041278);
\end{tikzpicture}
\end{center}
\begin{center}
  \begin{tabular}{l | l}
$\delta = [IJK]-[IH]$ & $ \rho = F'M$ \\
$\delta = 2nIJ-[IH]$ & $\rho = \tan i f$ \\
$\delta = 2n\frac{e}{\cos i}-e\frac{\sin^2i}{\cos i}$ & Or $i$ est faible \\
$\delta = 2ne\cos i$ & $ \Rightarrow \tan i \simeq i \Rightarrow \rho = if$ \\
  \end{tabular}
\end{center}

La différence de marce entre les deux rayons parallèles interférants dans le plan focal d'une lentillle convergeante et faisant un angle $i$ faible par rapport à la normale au miroir, donc :

\[\delta = 2ne\cos i \Rightarrow \delta \simeq 2ne \left( 1-\frac{i^2}{2}\right) \Rightarrow \delta \simeq 2ne \left( 1-\frac{\rho^2}{2f^2}\right)\]

Lorsque l'on observe les rayons paralèles dans le plan focal d'une lentille convergeante de distance focal $f$, les franges d'interférences sont des anneaux concentriques centrés sur l'axe optique de la lentille dont le rayon $\rho$ est tel que $\rho = if$.

L'ordre d'interférence $P$ est tel que :     
\[ P = \frac{\delta}{\lambda_0} = \frac{2ne}{\lambda_0}\left( 1-\frac{\rho^2}{2f^2}\right)\]


L'ordre d'interférence est maximal au centre $F'$ pour $i = 0 \Rightarrow \rho =0$ et tel que $P_0 = \frac{2ne}{\lambda_0}$

\[\frac{2ne}{\lambda_0} - P = \frac{2ne}{\lambda_0}\times\frac{\rho^2}{2f^2}\]
\[\rho = f\sqrt{\frac{\lambda_0}{ne}\left(\frac{2ne}{\lambda_0} - P\right)}\]
\[\rho = f\sqrt{\frac{2}{P_0}\left(\frac{2ne}{\lambda_0} - P\right)}\]
\[\rho = f\sqrt{\frac{2}{P_0}\left(P_0 - P\right)}\]

Les anneaux brillants ont des rayons $\rho_{mb}=f\sqrt{\frac{2}{P_0}\left(P_0 - m\right)} $ avec $m$ entier et $P=m$.

Les anneaux sombres ont des rayons $\rho_{ms}=f\sqrt{\frac{2}{P_0}\left(P_0 - \left(m +\frac{1}{2}\right)\right)} $ avec $m$ entier et $P=m +\frac{1}{2}$

Si le centre des anneaux est un point brillant alors $P = \frac{2ne}{\lambda_0} = m_0$ (entier). Avec $m_0$ la valeur maximale de $m$ entier.

\[\Rightarrow \rho_{mb}=f\sqrt{\frac{2}{m_0}\left(m_0 - m\right)}\]

Si le centre des anneaux est un point sombre alors $P = \frac{2ne}{\lambda_0} = m_0 + \frac{1}{2}$. 

\[\Rightarrow \rho_{ms}=f\sqrt{\frac{2}{P}\left(\left(m_0 + \frac{1}{2}\right) - \left(m + \frac{1}{2}\right)\right)}\]
Lorsque le centre des anneaux est un point brillant, le rayon des anneaux brillants varies comme la racine carré de nombre entier succesif $(m_0-m)$, les anneaux se resserent donc à mesure que l'on séloingne du centre.

A un anneau donné correspond un ordre d'interférence donné tel que  :

\[P = \frac{2ne \cos i}{\lambda_0}\]

Ainsi lorsque $e$ diminu, $\cos i$ augmente, donc $i$ diminu.

Les anneaux retrecissent donc et finissent par disparaitre au centre. Lorsque $e = 0$ il n'y a plus d'anneau ; l'intensité est uniforme, on réalise le contact optique et la peinte plate est obtenue.

Si on considère le cas général ou l'ordre d'interférence au centre, $P_0 = \frac{2ne}{\lambda_0 }$ n'est ni entier (brillant) ni demi-entier (sombre), on peut exprimer $P_0$ tel que $P_0 = m_0 + \varepsilon$ avec $0\leqslant \varepsilon\leqslant 1$ et $m_0$ entier. 
L'ordre d'interférence du premier anneau brillant : 
\begin{center}
\begin{tabular}{ r l}
  $1$\textsuperscript{er} : & $m_0 = p_0$ \\
  $2$\textsuperscript{ème} : & $m_0 - 1 = p_1$ \\
  $q$\textsuperscript{ème} : & $m_0 -(q - 1) = m_0 -q + 1 = p_{q-1}$
\end{tabular}
\end{center}

\[p_q = f\sqrt{\frac{\lambda_0}{ne}\left(\frac{2ne}{\lambda_0} - p_q\right)} = f\sqrt{\frac{\lambda_0}{ne}\left(p_0 - p_q\right)}\]
\[ \Leftrightarrow p_q = f\sqrt{\frac{\lambda_0}{ne}\left(m_0 + \varepsilon -m_0 +q -1\right)}\]
\[ \Leftrightarrow p_q = f\sqrt{\frac{\lambda_0}{ne}\left(\varepsilon +q -1\right)}\]

\begin{center}
\begin{tikzpicture}[line cap=round,line join=round,>=triangle 45,x=1.0cm,y=1.0cm]
\draw [color=cqcqcq,dash pattern=on 1pt off 1pt, xstep=0.5cm,ystep=0.5cm] (3.99,2.49) grid (7,5.5);
\clip(3.5,2.5) rectangle (7,5.5);
\draw [color=qqzzff] (5.5,4) circle (1.5cm);
\draw [color=qqzzff] (5.5,4) circle (1cm);
\draw [color=qqzzff] (5.5,4) circle (0.5cm);
\draw (5.36,4.47) node[anchor=north west] {$q = 1$};
\draw (5.35,5.01) node[anchor=north west] {$q = 2$};
\draw (5.36,5.51) node[anchor=north west] {$q = 3$};
\draw (5.85,4.17) node[anchor=north west] {$p = 3$};
\draw (6.34,4.17) node[anchor=north west] {$p = 2$};
\draw (6.84,4.18) node[anchor=north west] {$p = 1$};
\end{tikzpicture}
\end{center}

%%%%%%%%%%%%%%%%%%%%%%%%%%%%%%%%%%%%%%%%%%%%%%%%%%%%%%%%%%%%%%%%%%%%%%%%%%%%%%%%%%%%%%%%%%%%%%%%%%%%%%%%%%%%%%%%%%%%%
\section{Franges d'égales épaisseur : Franges de Fizeau}
\subsection{Cas général d'une lame quelconque}

Les expressions de la difference de marche totales (difference de marche géometrique et physique) pour les interferences observés en transmissions et en réflexion sur une lame mince d'indice $n$ sont :
\[
\delta_r = 2ne\cos r +\frac{\lambda_0}{r}\verb|   |;\verb|   |\delta_t = 2ne\cos r
\]\[
\varPhi_r = \frac{2\pi}{\lambda_0}\times 2ne \cos r + \pi\verb|   |;\verb|   |\varPhi_t = \frac{2\pi}{\lambda_0}\times 2ne \cos r
\]

L'intensité des ondes interférents en un point de la surface de localisation des franges est donné par :
\[ I = I_1 +I_2+2\sqrt{I_1I_2}cos\Theta  cos \Phi\]

Avec $\cos\Theta = \cos\varPhi = 1$ pour des rayons incidents proche de la normal au dioptre de la lame en coin.

Les points $M$ d'observation des interférences d'égales intensité sont donc tel que $e=cte$. Les franges obtenues sont donc apelées "Franges d'égales épaisseur" puisque les franges d'égale intensité sont obtenue à partir d'ondes  refléchies ou transmise par une même épaisseur de lame. 

Les franges observés d'égales intensité d'une lame mince pour la transmission sont telles que :

\[p+1 = \frac{2n(e+\Delta e)}{\lambda_0} => \Delta_e=\frac{\lambda_0}{2n}\]

Et en reflexion : 

\[p+1 = \frac{2n(e+\Delta e)}{\lambda_0} => \Delta_e=\frac{\lambda_0}{2n}\]



Les franges dégales intensité en reflexion ou en transmission sont définies par les intersections de la surface de la lame avec des plans paralèlle équidistant de $\frac{\lambda0}{2n}$.
\subsection{Lame en coin}

Soit $AB$ la distance entre 2 franges d'égales intensité lumineuse contigue à la surface d'une lame en coin. On a :
\[\Delta_e = AB\sin \alpha\]
Avec $\alpha$ très petit entre la surface des deux dioptres :
\[\Delta_e = AB \alpha\]
Compte tenu de la relation entre $\Delta_e$ on à :
\[AB = \frac{\lambda_0}{2n\alpha}\]
 $AB$ est constante, on peut donc parler de constante entre deux franges d'égales intensité lumineuse contigue avec : 

 \[I_f = \frac{\lambda_0}{2n\alpha}\]
 
$e$ est environ égal à $\alpha X$\\ Avec X la distance entre la frange et l'arrête du coin. L'ensemble des points d'égales intensité obtenue pour e = cte sont donc tels que $X = cte$. Les franges sont donc des droites parallèles à l'arrête du coin.
\subsection{Coin d'air}

Il existe 2 méthodes pour obtenir des interférences par un coin d'air :

- Dispositif des anneaux de Newton\\
- Interféromètre de Michelson réglé en coin d'air 

\subsubsection{Dispositif des anneaux de Newton}

La lame d'aire est réalisée entre une lentille plan convexe dont la face convexe est placée en regard d'une lame à face parallèle. 
Soit $e$ l'épaisseur de la lame au point considéré au poinr $M$ et $R$ le rayon de courbure de la lentille. Soit $\rho$ la distance de l'axe optique à la lentille. Les points $M$ d'égale intensité sont tel que $e = cte$. Les franges d'égale épaisseur déffinissent des anneaux concentriques du fait de la symétrie de revolution du système. En incidence normale on a : 

\[
\varPhi = \frac{2\pi}{\lambda_0}(2e+\delta_p)\]\[
e = e' + e_0\]\[
e'= ?\]\[\
R^2 =(R-e')^2 + \rho^2\]\[
\rho^2 = e'(2R - e') \simeq 2Re'\]\[
\varPhi = \frac{2\pi}{\lambda_0}\left(2e_0 + \frac{\rho^2}{R} + \delta_p\right)\]

Ansi le déphasage $\varPhi$ ne dépend que de $\rho$ la distance de $M$ à l'axe optique de la lentille. Les points d'égale intensité lumineuse déffinissent bien des anneaux centré sur l'axe optique de la lentille de rayon $\rho$.

Considéront le cas où l'on observe la figure d'interférence en réflexion : 
\[\delta_r = 2e + \frac{\lambda_0}{2}\]
\[\Rightarrow p = \frac{2e}{\lambda_0} + \frac{1}{2} = \frac{2e_0}{\lambda_0} + \frac{\rho^2}{\lambda_0R} + \frac{1}{2}\]
Sur l'axe de la lentille, au centre des anneaux $\rho = 0$ et donc :\[p_0= \frac{2e_0}{\lambda_0} + \frac{1}{2}\] \[\Rightarrow p - p_0 = \frac{\rho^2}{\lambda_0R} \Rightarrow p = \sqrt{\lambda_0 R (p-p_0)}\]

Cette formule reste valable pour le rayon $\rho$ des anneau observé en transmission.

Les anneaux brillants de Newton ont des rayons tel que $\rho_{mb} = \sqrt{\lambda_0 R(m-p_0)} $ Avec $m$ entier.

Les anneaux sombres de Newton ont des rayons tels que $\rho_{ms} = \sqrt{\lambda_0 R(m+ \frac{1}{2}-p_0)} $ Avec $m$ entier et $p = m +\frac{1}{2}$. On peut exprimer $p_0$ dans le cas général : $p_0 = m_0 + \varepsilon$ avec $0\leqslant\varepsilon< 1$ et $m_0$ la valeur minimum de $m$ entier.

Le rayon du Q ième anneau brillant est égal à $\sqrt{\lambda_0  R (p_q-p_0)}$

Si la lentille plan convexe repose directement sur la lame de verre $e_0 =0$ et on en déduit $\rho_m = \sqrt{\lambda_0 R m} $ (point sombre au centre.

Ansi on constate que le rayon des anneau sombre obtenue par reflexion varrie comme la racine carré d'entier successif.De plus $m = m_0 = 0$ donc le centre est un point sombre. L'ordre d'interférence est $p_0 = \frac{1}{2}$.

A un anneau donné correspond à un ordre d'interférence donné ainsi lorsque $e$ diminue c'est à dire si $e_0$ diminue alors $\rho$ augmente. Les anneaux ont un rayon qui augmente avec la diminution de $e$.

\subsubsection{Interphéromètre de Michelson : utilisation en coin d'air}

Soit une source $S$ étendue éclairant le Michelson à partir de la position de contact optique où $M_1$ est le symétrique de de $M_2$ par rapport à l'axe de la séparatrice, on incline le mirroir $M_2$ d'un angle $\alpha$ très petit. Comme $\alpha$ est très petit on peut considérer la figure d'interférence comme étant dans un plan : le plan des mirroirs.\\
Ainsi compte tenu de la localisation de la figure d'interférence dans le plan des mirroirs, celles ci pourrons être observées soit directement à l'oeil nu en l'accomodant dans ce plan soit en formant l'image du plan des mirroirs par un lentille mince convergente (observation dans le plan conjugué des mirroirs et non dans le plan focal image de la lentille). Au voisinage des mirroirs, la différence de marche entre les rayons interférents voit approximativement $\delta = 2e$. Or l'épaisseur de la lame dépend uniquement du point observé et de sa distance $X$ par rapport à l'intersection des mirroirs. En effet, on a $e \simeq \alpha x$ (pour $\alpha$ très petit) et l'interfrange $i_f= \lambda_0/\alpha$ (avec $\delta \simeq 2\alpha X$).
\end{document}
